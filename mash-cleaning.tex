\documentclass[a4paper,10pt]{article}
\usepackage[utf8]{inputenc}
\usepackage{graphicx}

%http://tex.stackexchange.com/questions/17489/change-caption-name-of-figures
\renewcommand{\figurename}{Step}

%opening
\title{Mash Cleaning}
\author{Ryan Duve}

\begin{document}

\maketitle

\begin{abstract}

\fbox{\parbox{.75\textwidth}{Only workers highly familiar with the technical aspects of Hifrost are authorized to perform this procedure.
}}
\vspace{.5cm}


The mash that came from Germany has an unknown amount of air or water vapor in it.  We will clean it according the following procedure.

This document assumes prior knowledge of the Hifrost pumping system and vacuum systems and components in general.  Please consult the Hifrost manual for context of this procedure and definitions/explanations of the system.
\end{abstract}

\section{Prep}
\begin{figure}[htbp!]
 \centering
 \includegraphics[width=\textwidth]{./mash-cleaning-schematic-1.png}
 % he-3-transfer-04-evac-system-with-p7.png: 0x0 pixel, 0dpi, 0.00x0.00 cm, bb=
 \caption{Hook up system.  The relevant part of the vacuum system; red channels indicate the mash transport path from T3 to the external tank.}
 \label{a}
\end{figure}

\begin{itemize}
 \item Configure the system as shown in Figure \ref{a}. Purge all lines along the transport path (keeping HVT3 closed!) with P7 to clean the system.
 \item Leak check the external tank and new connections (e.g., blanked condensor line).
 \item Close all valves; kill P7; fill trap with LN$_2$
\end{itemize}

\section{Cleaning}

\begin{figure}[htbp!]
 \centering
 \includegraphics[width=\textwidth]{./mash-cleaning-schematic-2-gas-to-ext-tank.png}
 % he-3-transfer-04-evac-system-with-p7.png: 0x0 pixel, 0dpi, 0.00x0.00 cm, bb=
 \caption{Open valves in this order: HVT3, HV19, HV20, HV3, HV4, HV5, HV14, HV13, HV12, HV9.  Wait until external tank equilibrates with T3.}
 \label{b}
\end{figure}

\begin{figure}[htbp!]
 \centering
 \includegraphics[width=\textwidth]{./mash-cleaning-schematic-3-pump-gas-to-ext-tank.png}
 % he-3-transfer-04-evac-system-with-p7.png: 0x0 pixel, 0dpi, 0.00x0.00 cm, bb=
 \caption{Close HV19, HV20; then open HV17 and turn on P10.  Slowly open HV19 as MT3 drops.}
 \label{c}
\end{figure}


\begin{figure}[htbp!]
 \centering
 \includegraphics[width=\textwidth]{./mash-cleaning-schematic-4-gas-to-t3.png}
 % he-3-transfer-04-evac-system-with-p7.png: 0x0 pixel, 0dpi, 0.00x0.00 cm, bb=
 \caption{When MT3 is around 0 mbar, close HV19, then kill P10.  Close HV17.  Open HV20 and HV19.  Wait until T3 and external tank equilibrate.}
 \label{d}
\end{figure}


\begin{figure}[htbp!]
 \centering
 \includegraphics[width=\textwidth]{./mash-cleaning-schematic-5-pump-gas-to-t3.png}
 % he-3-transfer-04-evac-system-with-p7.png: 0x0 pixel, 0dpi, 0.00x0.00 cm, bb=
 \caption{Close HV19 and HV20.  Open HV18 and then start P10.  Slowly open HV20; wait until external tank is empty.}
 \label{e}
\end{figure}

\begin{figure}[htbp!]
 \centering
 \includegraphics[width=\textwidth]{./mash-cleaning-schematic-6-repeat.png}
 % he-3-transfer-04-evac-system-with-p7.png: 0x0 pixel, 0dpi, 0.00x0.00 cm, bb=
 \caption{Repeat Steps 2-5 until MT3 remains constant between passes.}
 \label{f}
\end{figure}


\begin{figure}[htbp!]
 \centering
 \includegraphics[width=\textwidth]{./mash-cleaning-schematic-7-measure-trapped-gas}
 % he-3-transfer-04-evac-system-with-p7.png: 0x0 pixel, 0dpi, 0.00x0.00 cm, bb=
 \caption{Close all valves, then open HV4, HV3, HV20, HV19 and HV1.  Warm up trap and record pressure M5 to estimate how much air was trapped.}
 \label{g}
\end{figure}


\end{document}
